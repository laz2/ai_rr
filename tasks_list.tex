
\begin{enumerate}
\item Определить вид графа:
  \begin{enumerate}[1.]
  \item (1) Дерево (нг, ог)
  \item (1) Ациклический граф (нг, ог)
  \item (1) Полуэйлеров/эйлеров граф (нг)
  \item (3) Гамильтонов граф (нг)
  \item (0) Полный граф (нг)
  \item (2) Связный граф (нг)
  \item (3) Сильно-связный граф (ог)
  \item (2) Двусвязный граф 
  \item (2) Двудольный неориентированный граф  
  \item (2) Регулярный, реберно-регулярныйнеориентированный граф  
  \item (2) Симметричный, антисимметричный, частично симметричный орграф 
  \item (2) Транзитивный, антитранзитивный, частично транзитивный орграф 
  \item (2) Рефлексивный, антирефлексивный, частично рефлексивныйорграф 
  \item (2) Функциональный, контрафункциональныйорграф 
  \item (1) Вычерчиваемый граф 
  \item (2) Односторонне связный орграф  
  \item (4) Кактус 
  \item (1) Турнир, транзитивный турнир (ог)
  \item (1) Граф Бержа (ог)
  \item (3) Граф Паппа (нг)
  \item (4) Планарный граф (нг, ог)
  \item (3) Транзитируемый граф (нг)
  \end{enumerate}
\item Задачи на определение числовых характеристик графа:
  \begin{enumerate}[1.]
  \item (2) Радиус (нг, ог, внг, вог)
  \item (2) Диаметр (нг, ог, внг, вог)
  \item (2) Средний диаметр (нг, ог, внг, вог)
  \item (1) Полустепени захода/исхода и средниеполустепенивсех вершин в орграфе 
  \item (1) Минимальную степень/среднюю степень/максимальную степень ребра в неориентированном графе  
  \item (4) Число вершинной связности (нг, ог)
  \item (4) Число рёберной связности (нг, ог)
  \item (2) Среднее и максимальное расстояние между центральными вершинаминеориентированного графа  
  \item (3) Число хорд неориентированного графа 
  \item (2) Минимальное и среднее расстояние между периферийными вершинами неориентированного графа  
  \item (0) Цикломатическое число неориентированного графа  
  \item (3) Окружение орграфа  
  \item (3) Обхват орграфа  
  \item (2) Число компонентов связности неориентированного графа  
  \item (3) Число Хадвигера для неориентированного графа 
  \item (5) Определить толщину неориентированного графа 
  \item (4) Индекс компонент относительно простой цепи в неориентированном графе 
  \end{enumerate}
\item Задачи на построение графовых структур:
  \begin{enumerate}[1.]
  \item (4) Сгенерировать клетку указанного обхвата  
  \item (?) Найти граф, являющийся пересечением множества всех диаметральных цепей неориентированного графа  
  \item (?) Найти граф, являющийся объединением всех радиусов графа  
  \item (?) Найти граф, являющийся объединеним множества всех диаметров графа  
  \item (?) Найти граф, являющийся пересечением множества всех гамильтоновых циклов графа  
  \item (?) Найти граф, являющийся объединением множества всех гамильтоновых циклов графа  
  \end{enumerate}
\item Операции над графами:
  \begin{enumerate}[1.]
  \item (2) Найти декартово произведение двух неориентированных графов 
  \item (2) Найти декартову суммудвух неориентированных графов 
  \item (2) Найти прямое (тензорное) произведение двух неориентированных графов 
  \item (2) Найти сильное произведение двух неориентированных графов 
  \item (2) Найти композицию двух неориентированных графов 
  \item (2) Найти модульное произведение двух неориентированных графов 
  \item (2) Найти большое модульное произведение двух неориентированных графов 
  \item (2) Найти объединение множества неориентированных графов 
  \item (2) Найти пересечение множества неориентированных графов 
  \item (2) Найти дополнение и фактор-дополнение неориентированного графа 
  \item (2) Найти граф инциденций неориентированного графа 
  \item (2) Найти реберный граф для неориентированного графа 
  \item (2) Найти граф смежностей для неориентированного графа 
  \item (2) Найти тотальный граф для неориентированного графа 
  \item (2) Найти граф замыкания неориентированного графа  
  \item (4) Найти граф конденсации для орграфа 
  \item (4) Найти граф каркасов для неориентированного графа 
  \end{enumerate}
\item Задачи на поиск в графе:
  \begin{enumerate}[1.]
  \item (2) Определить эксцентриситет каждой вершины в неориентированном графе 
  \item (3) Найти эйлеров цикл в графе (нг, ог)
  \item (3) Найти гамильтонов цикл (нг, ог)
  \item (3) Найти компоненты связности в неориентированном графе 
  \item (3) Найти сильные компоненты связности в орграфе  
  \item (0) Найти вершины с указанной степенью 
  \item (3) Найти минимальный остов в неориентированном графа  
  \item (3) Найти доли неориентированного графа  
  \item (0) Найти тупики/антитупикив ориентированныйграфе 
  \item (3) Найти максимальный простой разрез (нг, ог, внг, вог)
  \item (3) Найти минимальный простой разрез (нг, ог, внг, вог)
  \item (?) Найти все пары вершин с указанным расстоянием (нг, ог, внг, вог)
  \item (?) Найти множество рёбер, удаление которых приводит к увеличению числа компонентов связности ориентированного графа  
  \item (3) Найти точки сочленения неориентированного графа 
  \item (3) Найти мосты в неориентированном графе 
  \item (5) Найти множество вершин, удаление которых приводит к увеличению числа компонентов связности ориентированного графа  
  \item (3) Найти циклы указанной длины (нг, ог, внг, вог)
  \item (3) Найти максимальный путь между заданными вершинами (нг, ог, внг, вог)
  \item (?) Нахождение критических путей во взвешенном неориентированном графе  
  \item (?) Найти множество вершин, удаление которых приводит к увеличению числа компонентов связности неориентированного графа  
  \item (3) Найти простые цепи указанной длины (нг, ог, внг, вог)
  \item (?) Найти вершины с указанной полустепенью 
  \item (1) Найти все доминирующие вершины (нг, ог, внг, вог)
  \item (?) Найти центры графа  
  \item (?) Найти рёбра с указанной степенью  
  \item (2) Найти все периферийные вершины (нг, ог, внг, вог)
  \item (?) Найти множество рёбер, удаление которых приводит к увеличению числа компонентов связности неориентированного графа  
  \item (?) Найти звёзды с заданным числом листьев  
  \item (4) Найти дерево кратчайших путей (нг, ог, внг, вог)
  \item (?) Определить рёбра и их степени в неориентированноммультиграфе 
  \item (?) Найти n-фактор для указанного графа  
  \item (5) Поиск подграфов в неориентированном графе, изоморфных графу-образцу 
  \item (3) Сформировать множество различных суграфов неориентированного графа 
  \item (3) Сформировать множество различных подграфов неориентированного графа 
  \end{enumerate}
\item Приведение графа к указанному виду:
  \begin{enumerate}[1.]
  \item (3) Найти минимальное множество вершин неориентированного графа, удаление которых позволяет сделать его деревом 
  \item (3) Найти минимальное множество рёбернеориентированного графа, удаление которых позволяет сделать его деревом 
  \item (5) Найти минимальное множество вершин неориентированного графа, удаление которых позволяет сделать его планарным 
  \item (5) Найти минимальное множество рёбер графа, удаление которых позволяет сделать его планарным 
  \end{enumerate}
\end{enumerate}

%%% Local Variables:
%%% mode: latex
%%% TeX-master: "ai_rr"
%%% End:
