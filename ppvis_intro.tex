
\phantomsection
\addcontentsline{toc}{section}{Введение}
\section*{Введение}
\label{sec:ppvis_intro}

Расчетная работа по курсу ППвИС является продолжением расчетной работы
по курсу ОИИ и основывается на предыдущих результатах. Теперь вам
необходимо будет адаптировать разработанный алгоритм к решению
теоретико-графовой задачи в семантической памяти, выполнив следующие
этапы:

\begin{itemize}
\item разработка программы решения теоретико-графовой задачи в
  семантической памяти на языке программирования C++
  (см. раздел~\ref{cha:libscalgo})
\item разработка программы решения теоретико-графовой задачи на языке
  программирования, предназначенном для обработки семантических сетей
  (см. раздел~\ref{cha:scpalgo})
\end{itemize}

Все используемые в этой части материалы, исходные тексты, установочные
файлы могут быть найдены на кафедральном информационном
сервере по следующему пути:

{\footnotesize
\begin{verbatim}
\\Info\StudInfo\~Методическое обеспечение кафедры\~Учебные курсы\2 курс\ППвИС\1sem\Расчётная работа\
\end{verbatim}
}

%%% Local Variables: 
%%% mode: latex
%%% TeX-master: "ai_rr"
%%% End: 
